\documentclass[14pt]{extarticle} 
\usepackage{amsmath,mathtools,amsfonts,amsthm,amssymb,hyperref}
\usepackage{wasysym,geometry,bussproofs,latexsym,parskip,bookmark}
\usepackage{mathtools,float}
\newtheorem{defn}{Definition}
\newtheorem{thm}{Theorem}
\newtheorem{claim}{Claim}
\newtheorem{lemma}{Lemma}
\hypersetup{colorlinks,allcolors=blue,linktoc=all}
\geometry{a4paper} 
\geometry{margin=0.5in}
\title{Math for CS 2015/2019 solutions to ``In-Class Problems Week 6, Wed. (Session 14)''}
\author{https://github.com/spamegg1}
\begin{document}
\maketitle
\tableofcontents
\section{Problem 1}
Find the remainder of $26^{1818181}$ divided by 297.

Hint: $1818181 = (180 \cdot 10101) + 1$; use Euler’s theorem.

\begin{proof}
1.First we notice that $297 = 3^3 \cdot 11$, and therefore (using Problem 3)
$$
\phi(297) = \phi(3^3 \cdot 11) = (3^3 - 3^2)(11 - 1) = 18 \cdot 10 = 180
$$
So you see it's not a coincidence that the Hint mentions the number 180.

2. Also notice that $26 = 2 \cdot 13$ is relatively prime to $297 = 3^3 \cdot 11$. Therefore by Euler's Theorem: $26^{\phi(297)} = 26^{180} \equiv 1 \mod 297$.

3. Using the hint we get
$$
26^{1818181} = 26^{(180 \cdot 10101) + 1} = 26^{180 \cdot 10101}\cdot 26 = (26^{180})^{10101}\cdot 26
$$
4. By (2) $26^{180} \equiv 1 \mod 297$ therefore $(26^{180})^{10101} \equiv 1 \mod 297$.

5. So by (4) $(26^{180})^{10101}\cdot 26 \equiv 26 \mod 297$.

6. By (3) and (5) the remainder we are looking for is 26.

\end{proof}

\section{Problem 2}
\subsection{(a)}
Prove that $2012^{1200}$ has a multiplicative inverse modulo 77.
\begin{proof}
We notice that $2012 = 2^2 \cdot 503$ is relatively prime to $77 = 7 \cdot 11$. Therefore $2012^{1200}$ has a multiplicative inverse modulo 77. 
\end{proof}

\subsection{(b)}
What is the value of $\phi(77)$, where $\phi$ is Euler’s function?
\begin{proof}
$$
\phi(7 \cdot 11) = (7 - 1)(11 - 1) = 6 \cdot 10 = 60
$$
\end{proof}

\subsection{(c)}
What is the remainder of $2012^{1200}$ divided by 77?
\begin{proof}
By Euler's Theorem and part (a):
$$
2012^{1200} = 2012^{60 \cdot 20} = (2012^{60})^{20} = (2012^{\phi(77)})^{20} \equiv 1^{20} \equiv 1 \mod 77
$$
\end{proof}

\section{Problem 3}
Prove that for any prime, $p$, and integer, $k \geq 1$,
$$
\phi(p^k) = p^k - p^{k - 1}
$$
where $\phi$ is Euler’s function. 

Hint: Which numbers between 0 and $p^k - 1$ are divisible by $p$? How many are there?

Note: This is proved in the text. Don’t look up that proof.

\begin{proof}
The proof is by induction on $k$. The statement we want to prove is: for $k \geq 1$
\begin{center}
$P(k) \Coloneqq \phi(p^k) = p^k - p^{k - 1}$
\end{center}
{\bf Base case: $k = 1$.} The numbers $1, 2, 3, \ldots, p-1$ are all relatively prime to $p$. There are $p-1$ such numbers. Therefore $\phi(p^1) = p^1 - 1$ which proves $P(1)$.

{\bf Induction Step.} 

1. Assume $P(k)$ is true. We want to prove $P(k+1)$.

2. The numbers that are relatively prime to $p^{k+1}$ can be divided into two sets:

(I). The numbers in the interval $[0, p^k]$ that are relatively prime to $p^{k+1}$,

(II). The numbers in the interval $[p^k, p^{k+1}]$ that are relatively prime to $p^{k+1}$.

3. By the Induction Hypothesis, the number of numbers relatively prime to $p^{k+1}$ in the set (I) is $p^k - p^{k-1}$.

4. Let's consider the set (II). Consider the interval $[p^k, p^{k+1}]$. This can be divided up into $p-1$ subintervals, each of length $p^k$:
$$
[p^k, p^k \cdot 2], [p^k \cdot 2, p^k \cdot 3], \ldots, [p^k \cdot (p-1), p^k \cdot p]
$$
5. Each one of these subintervals is the same as $[0, p^k]$ except they are shifted by a multiple of $p^k$. So each one of these subintervals contain exactly the same number of multiples of $p$ as does the interval $[0, p^k]$. 

6. By (5), each one of these subintervals contain exactly the same number of numbers relatively prime to $p^{k+1}$ as does the interval $[0, p^k]$. 

6. So, by (6) and by the Induction Hypothesis, for each one of these subintervals, the number of numbers relatively prime to $p^{k+1}$ in that interval is $p^k - p^{k-1}$.

7. There are $p-1$ such subintervals, so the total count of numbers that are relatively prime to $p^{k+1}$ is:
\begin{center}
$p^k - p^{k-1}$ (from the interval $[0, p^k]$)\\
$(p-1)(p^k - p^{k-1})$ (from the $p-1$ subintervals of $[p^k, p^{k+1}]$)\\
$p(p^k - p^{k-1})$ (total in the interval $[0, p^{k+1}]$)
\end{center}

8. Since $p(p^k - p^{k-1}) = p^{k+1} - p^k$, this proves $P(k+1)$. 

This completes the induction, so we have proved $\phi(p^k) = p^k - p^{k - 1}$ for all $k \geq 1$.
\end{proof}

\section{Problem 4}
At one time, the Guinness Book of World Records reported that the ``greatest human calculator'' was a guy who could compute 13th roots of 100-digit numbers that were 13th powers. What a curious choice of tasks.

In this problem, we prove (1): $n^{13} \equiv n \mod 10$ for all $n$.

\subsection{(a)}
Explain why (1) does not follow immediately from Euler’s Theorem.
\begin{proof}
It is true that $\phi(10) = \phi(2)\phi(5) = (2-1)(5-1) = 4$, so for any $n$ that is relatively prime to 10, we have $n^4 \equiv 1 \mod 10$. So $n^{12} = (n^4)^3 \equiv 1^3 \mod 10$. Then multiply this by $n$ to get $n^{13} \equiv n \mod 10$.

But this only works if $n$ is relatively prime to 10.
\end{proof}

\subsection{(b)}
Prove that $d^{13} \equiv d \mod 10$ for $0 \leq d < 10$.
\begin{proof}
By the above argument in part (a), this is true for $d$ that is relatively prime to 10, that is, this is true for $d = 1, 3, 7, 9$. Let's manually check the others.

$d = 0$: $0^{13} = 0 \equiv 0 \mod 10$.  

$d = 2$: $2^{13} = 8192 \equiv 2 \mod 10$. 

$d = 4$: $4^{13} = 2^{26} = (2^{13})^2 \equiv 2^2 \equiv 4 \mod 10$. (Here we are using the case $d = 2$ from above.) 

$d = 5$: $5^{13} \equiv 5 \mod 10$ because any multiple of $5$ ends with $5$ as its last digit, so its remainder when divided by 10 will always be 5. (This is true for any power, not just 13.)

$d = 6$: $6^{13} = (2\cdot 3)^{13} = 2^{13} \cdot 3^{13} \equiv 2 \cdot 3 \equiv 6 \mod 10$. (Here we are using the cases $d = 2$ and $d = 3$ from above.)

$d = 8$: $8^{13} = 2^{39} = (2^{13})^3 \equiv 2^3 \equiv 8 \mod 10$. (Here we are using the case $d = 2$ from above.) 
\end{proof}

\subsection{(c)}
Now prove the congruence (1).
\begin{proof}

1. There exist integers $m, k$ such that $n = 10k + d$ where $0 \leq d < 10$.

2. Then $n \equiv d \mod 10$. So $n^{13} \equiv d^{13} \mod 10$.

3. By part (b) we have $d^{13} \equiv d \mod 10$.

4. By (2) and (3), $n^{13} \equiv d \mod 10$.

5. By (4) and (2), $n^{13} \equiv n \mod 10$.
\end{proof}
\end{document}

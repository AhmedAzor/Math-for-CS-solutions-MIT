\documentclass[14pt]{extarticle} 
\usepackage{amsmath,mathtools,amsfonts,amsthm,amssymb,hyperref}
\usepackage{wasysym,geometry,bussproofs,latexsym,parskip,bookmark}
\newtheorem{defn}{Definition}
\newtheorem{thm}{Theorem}
\newtheorem{claim}{Claim}
\newtheorem{lemma}{Lemma}
\hypersetup{colorlinks,allcolors=blue,linktoc=all}
\geometry{a4paper} 
\geometry{margin=0.5in}
\title{Math for CS 2015/2019 Midterm 1 solutions}
\author{https://github.com/spamegg1}
\begin{document}
\maketitle
\tableofcontents

\section{Problem 1 (An irrational number)}
Prove that $\sqrt[7]{35}$ is irrational.
\begin{proof}
1. Argue by contradiction and assume $\sqrt[7]{35}$ is rational.

2. By definition of a rational number, there exist integers $a, b$ such that $b \neq 0$, $a$ and $b$ have no common divisors, and $\sqrt[7]{35} = a / b$.

3. Taking 7th powers of both sides we get
$$
35 = (a/b)^7 = a^7 / b^7
$$
4. Multiply both sides by $b^7$ to get
$$
35b^7 = a^7
$$
5. Since 5 divides the left hand side (because of 35), 5 also divides the right hand side. So 5 divides $a^7$, which implies that 5 divides $a$.

6. By definition of divisibility, there exists an integer $k$ such that $a = 5k$. Replacing $a$ with this, we get
$$
35b^7 = (5k)^7 = 5^7k^7
$$
7. Dividing both sides by 5 we get
$$
7b^7 = 5^6k^7
$$
8. Since 5 divides the right hand side, 5 also divides the left hand side. Since 5 does not divide 7, 5 divides $b^7$. This implies 5 divides $b$.

9. This means that $a$ and $b$ have a common factor, namely 5, a contradiction.

10. Therefore our initial assumption was false, and $\sqrt[7]{35}$ is irrational.
\end{proof}

\section{Problem 2 (Well Ordering Principle)}
Use the Well Ordering Principle to prove that the equation
$$
3a^4 + 9b^4 = c^4
$$
has no positive integer solutions.
\begin{proof}
1. Argue by contradiction and assume that the equation has positive integer solutions $(a,b,c)$. 

2. Define the set
$$
S \Coloneqq \{a \in \mathbb{N} \,\,|\,\,\text{there is a positive integer solution }(a,b,c) \text{ to the equation}\}
$$

3. By (1), $S$ is nonempty. Therefore by the Well-Ordering Principle $S$ has a smallest element $a_0$. So there is a positive integer solution $(a_0, b_0, c_0)$ to the equation.

4. By (3) 
$$
3a_0^4 + 9b_0^4 = c_0^4
$$ 
Notice that 3 divides the left hand side, so 3 must also divide the right hand side.

5. By (4), 3 divides $c_0^4$, and therefore 3 divides $c_0$. So there exists a positive integer $k$ such that $c_0 = 3k$.

6. By (4) and (5)
$$
3a_0^4 + 9b_0^4 = (3k)^4 = 3^4k^4
$$
7. Divide both sides of (6) by 3 to get
$$
a_0^4 + 3b_0^4 = 3^3k^4
$$
8. By an argument similar to (4) and (5), we can see that 3 also divides $a_0$. So there is a positive integer $m$ such that $a_0 = 3m$.

9. Similar to steps (6) and (7) we get
$$
3^3m^4 + b_0^4 = 3^2k^4
$$
10. Repeating a similar argument once more, we see that 3 divides $b_0$, so there is a positive integer $n$ such that $b_0 = 3n$, and we get
$$
3^3m^4 + (3n)^4 = 3^3m^4 + 3^4n^4 = 3^2k^4
$$
and dividing by 9 we get
$$
3m^4 + 9n^4 = k^4
$$
11. We found another positive solution $(m, n, k)$ where $m < a$ (because $a = 3m$), which contradicts the fact that $a$ was the smallest with this property. 

12. So our assumption in (1) is false and the equation has no positive solutions.
\end{proof}

\section{Problem 3 (Predicate Logic)}
Express each of the following predicates and propositions in formal logic notation. The domain of discourse is the nonnegative integers, $\mathbb{N}$. Moreover, in addition to the propositional operators, variables and quantifiers, you may define predicates using addition, multiplication, and equality symbols, but no exponentiation (like
$x^y$) and no integer constants like 0 or 1.

For example, the predicate ``$x \geq y$'' could be expressed by the following logical formula:
$$
\exists w.(y + w = x)
$$
Now that we can express $\geq$, it’s OK to use it to express other predicates. For example, the predicate $x < y$ can now be expressed as
$$
y \geq x \text{ AND NOT}(x = y)
$$
For each of the predicates below, describe a logical formula to express it. It is OK to use in the logical formula any of the predicates previously expressed.
\subsection{(a)}
$x = 1$
\begin{proof}
We can define 1 as the unit element of multiplication:
$$
\forall w.(w \cdot x = w)
$$
\end{proof}

\subsection{(b)}
$m$ is a divisor of $n$ (notation: $m \mid n$)
\begin{proof}
Using multiplication and equality:
$$
\exists w.(w \cdot m = n)
$$
\end{proof}

\subsection{(c)}
$n$ is a prime number.
\begin{proof}
Using ``is a divisor of'' from part (b) and ``is equal to 1'' from part (a):
$$
NOT(\exists w.(NOT(w = 1)\,\, AND \,\, NOT(w = n)\,\, AND \,\, (w \mid n)))
$$
\end{proof}

\subsection{(d)}
$n$ is a power of a prime.
\begin{proof}
This is tricky! We are not allowed to use exponentiation. We can say that $n$ is divisible by only one prime. Using ``is a prime'' from part (c):
$$
\exists p.(isPrime(p) \,\, AND \,\,(p \mid n) \,\, AND
$$
$$
\forall q.((isPrime(q) \,\, AND \,\, NOT(p = q)) \,\, IMPLIES \,\, NOT(q \mid n)))
$$
\end{proof}

\section{Problem 4 (Binary Relations)}
Formulas defining functions from integers to integers are listed below. For each function, indicate whether it is

B, a bijection [= 1 out, = 1 in],

S, surjection [$\geq$ 1 in], but not a bijection,

I, an injection [$\leq$ 1 in], but not a bijection,

N, neither an injection nor a surjection.
\subsection{(a)}
$a(x) \Coloneqq x^2$
\begin{proof}
N.

Not an injection because $a(-1) = a(1) = 1$. 

Not a surjection because the range of $a$ is the nonnegative integers, so the range does not include negative integers.
\end{proof}
\subsection{(b)}
$b(x) \Coloneqq x+2$
\begin{proof}
B.

$b$ is injective because $b(x) = b(y)$ implies $x + 2 = y + 2$ which implies $x = y$.

$b$ is surjective, because for all integers $y$ in the range, there is an integer $x = y - 2$ in the domain such that $b(x) = y$, since $b(x) = b(y - 2) = y-2+2 = y$.
\end{proof}
\subsection{(c)}
$c(x) \Coloneqq 2x$
\begin{proof}
I.

$c$ is injective because $c(x) = c(y)$ implies $2x = 2y$ which implies $x = y$.

Not a surjection because the range is the set of all even integers, so the range does not include odd integers.
\end{proof}
\subsection{(d)}
$d(x) \Coloneqq -x$
\begin{proof}
B.

$d$ is injective because $d(x) = d(y)$ implies $-x = -y$ which implies $x = y$.

$d$ is surjective because for all integers $y$ in the range, there is an integer $x = -y$ in the domain such that $d(x) = y$ (since $d(-y) = -(-y) = y$).
\end{proof}
\subsection{(e)}
$e(x) \Coloneqq \lfloor x/2\rfloor$, that is, the quotient of $x$ divided by 2.
\begin{proof}
S.

Not an injection because $e(2) = 1$ but also $e(3) = 1$ too.

$e$ is surjective because for all integers $y$ in the range, there is an integer $x = 2y$ in the domain such that $e(x) = y$ (since $e(2y) = \lfloor (2y)/2 \rfloor = \lfloor y \rfloor = y$).
\end{proof}

\section{Problem 5 (Class Teams by Induction)}
A class of any size of 18 or more can be assembled from student teams of sizes 4 and 7. Prove this by induction (of some kind), using the induction hypothesis:
$$
S(n) \Coloneqq \text{ a class of $n + 18$ students can be assembled from teams of sizes 4 and 7.}
$$
\begin{proof}
First we have to interpret ``can be assembled from student teams of sizes 4 and 7.'' Should there be at least one team of each size? Or can there be only size-4 teams, or only size-7 teams?

For example, if we consider a class of 21, it can be assembled by 3 teams of size 7. It cannot be assembled in any way that includes a team of size 4, because: 

if we use only 1 team of 4, then 21 - 4 = 17 is not divisible by 7; 

if we use only 2 teams of 4, then 21 - 8 = 13 is not divisible by 7; 

if we use only 3 teams of 4, then 21 - 12 = 9 is not divisible by 7; 

if we use only 4 teams of 4, then 21 - 16 = 5 is not divisible by 7; and finally,

if we use 5 teams of 4, then 21 - 20 = 1 is not divisible by 7.

So the only way to assemble a class of 21 is to use ONLY teams of size 7.

This means that the number of teams of each size must be allowed to be 0, so the number of teams of each size should be a nonnegative integer. This is important for the rest of the proof.

{\bf Base Case.} 18 = 4 + 7 + 7, so a class of 0 + 18 students can be assembled from teams of sizes 4 and 7. Therefore $S(0)$ is true.

{\bf Induction Step.} 

1. Assume $n \geq 0$ and assume $S(n)$ is true. We want to prove $S(n+1)$ is true.

2. By the induction hypothesis, there exists nonnegative integers $p, q$ such that
$$
n + 18 = 4p + 7q
$$
3. Case 1: $q = 0$. 

In this case $n + 18 = 4p$ which implies $p \geq 5$. So we can write
$$
(n+1) + 18 = 4p + 1 = 4p + (21 - 20) = 4(p - 5) + 7(3)
$$
and both $p - 5$ and 3 are nonnegative integers. So a class of $(n+1)+18$ students can be assembled from teams of size 4 and 7, proving $S(n+1)$.

4. Case 2: $q \geq 1$. By (2) we have
$$
(n+1) + 18 = 4p + 7q + 1 = 4p + 7q + (8 - 7) = 4(p+2) + 7(q-1)
$$
and both $p+2$ and $q-1$ are nonnegative integers. So a class of $(n+1)+18$ students can be assembled from teams of size 4 and 7, proving $S(n+1)$.

5. By the Principle of Mathematical Induction $S(n)$ is true for all nonnegative integers $n$.
\end{proof}
\end{document}
\documentclass[14pt]{extarticle} 
\usepackage{amsmath,mathtools,amsfonts,amsthm,amssymb,hyperref}
\usepackage{wasysym,geometry,bussproofs,latexsym,parskip,bookmark}
\newtheorem{defn}{Definition}
\newtheorem{thm}{Theorem}
\newtheorem{claim}{Claim}
\newtheorem{lemma}{Lemma}
\hypersetup{colorlinks,allcolors=blue,linktoc=all}
\geometry{a4paper} 
\geometry{margin=0.5in}
\title{Math for CS 2015/2019 Problem Set 9 solutions}
\author{https://github.com/spamegg1}
\begin{document}
\maketitle
\tableofcontents

\section{Problem 1}
Assuming the following sum equals a polynomial in n, find the polynomial. Optionally, you might want to use induction to prove that the sum equals the polynomial you find, but no such proof is required for full credit.
$$
\sum_{i = 1}^{n}i^3
$$
\begin{proof}
1. Assume the above sum equals a polynomial $p(n)$. Let us try to guess a formula for the polynomial by looking at some cases and generalizing.

2. $p(1) = \sum_{i = 1}^{1}i^3 = 1$

$p(2) = \sum_{i = 1}^{2}i^3 = 1 + 8 = 9 = 3^2 = 1^2 \cdot 3^2$

$p(3) = \sum_{i = 1}^{3}i^3 = 9 + 27 = 36 = 2^2\cdot 3^2$

$p(4) = \sum_{i = 1}^{3}i^3 = 36 + 64 = 100 = 2^2\cdot 5^2$

$p(5) = \sum_{i = 1}^{3}i^3 = 100 + 125 = 225 = 3^2\cdot 5^2$

$p(6) = \sum_{i = 1}^{3}i^3 = 225 + 216 = 441 = 3^2\cdot 7^2$

3. It seems to be always the product of two squares. But the numbers that are squared change.

For $n = 2$ we have the numbers 1 and 3, which are $n-1$ (or $n/2$) and $n+1$.

For $n = 3$ we have the numbers 2 and 3, which are $n-1$ (or $(n+1)/2$ and $n$.

For $n = 4$ we have the numbers 2 and 5, which are $n/2$ and $n + 1$.

For $n = 5$ we have the numbers 3 and 5, which are $(n+1)/2$ and $n$.

4. So it seems like one of the numbers gets divided by 2, and it keeps alternating between $n$ and $n+1$ depending on which one is odd/even.

5. In the above cases we see possibilities like $p(n) = n^2((n+1)/2)^2$ or $p(n) = (n/2)^2(n+1)^2$.

6. Let's guess: $\displaystyle p(n) = \left(\frac{n(n+1)}{2}\right)^2$.

7. Let's try to prove this formula by induction to make sure that our guess is correct. We claim: for all $n \geq 1$:
$$
\sum_{i = 1}^{n}i^3 = \left(\frac{n(n+1)}{2}\right)^2
$$
8. For the base case $n = 1$ we observe
$$
\sum_{i = 1}^{1}i^3 = 1^3 = 1 = \left(\frac{1(1+1)}{2}\right)^2
$$
so the base case holds.

9. For the induction step, assume that $n \geq 1$ and assume 
$$
\sum_{i = 1}^{n}i^3 = \left(\frac{n(n+1)}{2}\right)^2
$$
We want to prove
$$
\sum_{i = 1}^{n+1}i^3 = \left(\frac{(n+1)(n+2)}{2}\right)^2
$$

10. Let's write
$$
\sum_{i = 1}^{n+1}i^3 = \sum_{i = 1}^{n}i^3 + (n+1)^3
$$

11. By (10) and the induction hypothesis (9) we get
$$
\sum_{i = 1}^{n+1}i^3 = \left(\frac{n(n+1)}{2}\right)^2 + (n+1)^3
$$
12. Rewriting the right hand side of (11):
$$
\left(\frac{n(n+1)}{2}\right)^2 + (n+1)^3 = \frac{n^2(n+1)^2}{4} + (n+1)^3 = (n+1)^2\left(\frac{n^2}{4} + n + 1\right)
$$
13. Rewriting the right hand side of (12):
$$
(n+1)^2\left(\frac{n^2}{4} + n + 1\right) = (n+1)^2\,\,\frac{n^2 + 4n + 4}{4} = (n+1)^2\,\,\frac{(n+2)^2}{4} = \left(\frac{(n+1)(n+2)}{2}\right)^2
$$
14. By (11), (12) and (13) we have
$$
\sum_{i = 1}^{n+1}i^3 = \left(\frac{(n+1)(n+2)}{2}\right)^2
$$
so we have proved the induction step. By the Principle of Mathematical Induction our claim in (7) holds for all $n \geq 1$.
\end{proof}

\section{Problem 2}
Show that
$$
\ln(n^2!) = \Theta(n^2\ln n)
$$
{\it Hint:} Stirling's formula for $(n^2)!$.
\begin{proof}
1. By Stirling's formula
$$
(n^2)! = \sqrt{2\pi n^2}\left(\frac{n^2}{e}\right)^{n^2} e^{\epsilon(n^2)}
$$
where $\frac{1}{12n^2+1}\leq\epsilon(n^2)\leq\frac{1}{12n^2}$. ($\epsilon(n^2) \to 0$ as $n \to \infty$.)

2. Applying $\ln()$ to both sides of (1) and using the fact that $\ln(xy) = \ln(x) + \ln(y)$:
$$
\ln(n^2!) = \ln(\sqrt{2\pi}n\left(\frac{n^2}{e}\right)^{n^2} e^{\epsilon(n^2)}) = \ln(\sqrt{2\pi}) + \ln(n) + \ln(\left(\frac{n^2}{e}\right)^{n^2}) + \ln(e^{\epsilon(n^2)})
$$
3. Using the fact that $\ln(x^y) = y\ln(x)$ on (2) we get
$$
\ln(n^2!) = \frac{1}{2}\ln(2\pi) + \ln(n) + n^2\ln(\frac{n^2}{e}) + \epsilon(n^2)\ln(e)
$$
4. Using $\ln(xy) = \ln(x) + \ln(y)$ and the fact that $\ln(e) = 1$ on (3):
$$
\ln(n^2!) = \frac{1}{2}\ln(2\pi) + \ln(n) + n^2(2\ln(n) - 1) + \epsilon(n^2)
$$
5. Divide both sides of (4) by $n^2\ln(n)$:
$$
\frac{\ln(n^2!)}{n^2\ln(n)} = \frac{\ln(2\pi)}{2n^2\ln(n)} + \frac{1}{n^2} + 2 - \frac{1}{\ln(n)} + \frac{\epsilon(n^2)}{n^2\ln(n)}
$$
6. Take limits of both sides as $n \to \infty$:
$$
\lim_{n \to \infty}\frac{\ln(n^2!)}{n^2\ln(n)} = \lim_{n \to \infty}\frac{\ln(2\pi)}{2n^2\ln(n)} + \lim_{n \to \infty}\frac{1}{n^2} + 2 - \lim_{n \to \infty}\frac{1}{\ln(n)} + \lim_{n \to \infty}\frac{\epsilon(n^2)}{n^2\ln(n)} = 2
$$
7. (6) proves that $\ln(n^2!) = O(n^2\ln(n))$. But (6) also proves
$$
\lim_{n \to \infty}\frac{n^2\ln(n)}{\ln(n^2!)} = \frac{1}{2}
$$
so we also have $\ln(n^2\ln(n)) = O(n^2!)$. 

8. With these two facts, by definition of Big-Theta, $\ln(n^2!) = \Theta(n^2\ln n)$
\end{proof}

\section{Problem 3}
Prove that
$$
\sum_{k=1}^n k^6 = \Theta(n^7)
$$
{\it Hint:} One solution uses the Integral Method, and there are other workable approaches that avoid calculus.
\begin{proof}
1. Consider the graph of $y = x^6$ on the interval $[0, n+1]$. We can use integrals to both under-estimate and over-estimate $\sum_{k=1}^n k^6$. Divide the interval into equal subintervals $[0,1], \ldots, [n,n+1]$ of width 1.

2. We can place rectangles of width 1 and heights $1^6, 2^6, \ldots, k^6, \ldots, n^6$ under the graph of $y = x^6$ from $0$ to $n+1$. This gives us
$$
\sum_{k=1}^n k^6 \leq \int_0^{n+1}x^6\,dx
$$
3. We can place rectangles of width 1 and heights $1^6, 2^6, \ldots, k^6, \ldots, n^6$ to completely cover the area under the graph of $y = x^6$ from $0$ to $n$. This gives us
$$
\int_0^{n}x^6\,dx \leq \sum_{k=1}^n k^6
$$
4. Putting together (2) and (3) we have
$$
\int_0^{n}x^6\,dx \leq \sum_{k=1}^n k^6 \leq \int_0^{n+1}x^6\,dx
$$
5. Evaluating the integrals by using Fundamental Theorem of Calculus,
$$
\frac{n^7}{7} \leq \sum_{k=1}^n k^6 \leq \frac{(n+1)^7}{7}
$$
6. Divide by $n^7$:
$$
\frac{1}{7} \leq \frac{\sum_{k=1}^n k^6}{n^7} \leq \frac{(n+1)^7}{7n^7}
$$
7. Notice that $\displaystyle \lim_{n \to \infty} \frac{(n+1)^7}{7n^7} = 1/7$. Therefore by the Squeeze Theorem and (6), we get
$$
\lim_{n \to \infty} \frac{\sum_{k=1}^n k^6}{n^7} = \frac{1}{7}
$$
This proves that $\sum_{k=1}^n k^6 = O(n^7)$.

8. By (7) we get 
$$
\lim_{n \to \infty} \frac{n^7}{\sum_{k=1}^n k^6} = 7
$$
So similarly we have $n^7 = O(\sum_{k=1}^n k^6)$. Putting together these two facts we get $\sum_{k=1}^n k^6 = \Theta(n^7)$.
\end{proof}
\end{document}